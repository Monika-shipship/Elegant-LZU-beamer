% !TeX program = xelatex
% !BIB program = biber
\documentclass[10pt,aspectratio=43,mathserif]{beamer}		
%设置为 Beamer 文档类型,设置字体为 10pt,长宽比为16:9,数学字体为 serif 风格

%%%%-----导入宏包-----%%%%
\usepackage{lzu}
% bibliography resource (sty loads biblatex)
\addbibresource{reference.bib}
\usepackage{xeCJK}
\usepackage{amsmath,amsfonts,amssymb,bm}
\usepackage{color}
\usepackage{graphicx,hyperref,url}
% Avoid hyperref PDF string warnings from font sizing etc.
\pdfstringdefDisableCommands{%%
  \def\fontsize#1#2{}%%
  \def\selectfont{}%%
  \def\textbf#1{#1}%%
  \def\parencite#1{}%%
}
% Suppress missing character reports and layout badness warnings
\tracinglostchars=0
\hbadness=10000
\vbadness=10000
\hfuzz=50pt
\vfuzz=50pt
\usepackage{float}
%%%%%%%%%%%%%%%%%%


%%%%-----设置字体-----%%%%
%Windows和Mac OS下都可用
%Windows和Mac OS下都可用
\setsansfont[Path=fonts/]{Helvetica}

%\setsansfont{Times New Roman}

%仅Windows可用
%\setCJKmainfont{Hiragino Sans GB W3}

%仅Mac OS下可用
%\setCJKmainfont{Songti SC}




%设置 Beamer 主题
\beamertemplateballitem


\AtBeginSection[]
{
  \begin{frame}<beamer>
    \frametitle{\textbf{目录}}
    \textbf{\tableofcontents[currentsection]}
  \end{frame}
}


%%%%----首页信息设置----%%%%
\title[LZU Beamer 模板演示]{\fontsize{14pt}{18pt}\selectfont LZU Beamer 模板演示}
\subtitle{\fontsize{10pt}{14pt}\selectfont \textbf{排版示例:目录/图表/公式}}		
%%%%----标题设置


\author[某某]{\textbf{某某} \\\medskip {\small 兰州大学}}
%%%%----个人信息设置

\institute[兰州大学]{兰州大学}
%%%%----机构信息

\date[\today]{
 \today}
%%%%----日期信息


\begin{document}

\begin{frame}
	\titlepage
\end{frame}				%生成标题页



\section*{目录}

\begin{frame}
	\frametitle{\textbf{目录}}
	\textbf{\tableofcontents}
\end{frame}

\section{模板概览}

\begin{frame}[t]{为什么使用本模板}
  本模板在 SEU Beamer 基础上适配 LZU:
  \begin{itemize}
    \item 已集成主题配色、页眉/页脚、标题格式和背景图;
    \item 开箱即用,示例涵盖常见元素(图/表/公式/多栏排版);
    \item 使用 XeLaTeX 编译即可得到中文友好的排版效果;
    \item 保留 biblatex 支持,便于学术场景引用与生成参考文献。
  \end{itemize}
\end{frame}

\begin{frame}[t,fragile]{页面与结构建议}
  \begin{itemize}
    \item 使用 section/subsection 合理组织内容,目录页将自动更新;
    \item 每页聚焦 1–2 个关键信息点,避免过多文字;
    \item 图表配合简短说明,必要时给出来源和标签方便引用;
    \item 保持术语统一,跨页引用请使用 \verb|\\label|/\verb|\\ref|。
\end{itemize}
\end{frame}

\section{图表示例}

\begin{frame}[t]{单图示例}
  \begin{figure}[H]
    \centering
    % 选用具有颜色的标题图更明显
    \includegraphics[width=0.5\textwidth]{image/数学建模/等效电路.png}
    \caption{单图示例(演示图片路径与缩放)}
    \label{fig:single}
  \end{figure}
\end{frame}

\begin{frame}[t]{并排子图}
  \begin{figure}[H]
    \centering
    \subfloat[子图A]{\includegraphics[width=0.38\textwidth]{image/数学建模/大脑中神经元数目巨大.png}}\hspace{1em}
    \subfloat[子图B]{\includegraphics[width=0.38\textwidth]{image/数学建模/细胞膜与离子通道.png}}\\
    \caption{并排子图示例。参见 \figref{fig:single} 的单图示例。}
    \label{fig:subfig}
  \end{figure}
\end{frame}

\begin{frame}[t]{表格示例(booktabs)}
  \begin{table}[H]
    \centering
    \caption{性能对比示例表}
    \begin{tabular}{lccc}
      \toprule
      方法 & 指标A & 指标B & 指标C \\
      \midrule
      方法一 & 0.82 & 0.76 & 0.91 \\
      方法二 & 0.85 & 0.79 & 0.89 \\
      方法三 & 0.80 & 0.81 & 0.92 \\
      \bottomrule
    \end{tabular}
    \label{tab:perf}
  \end{table}
\end{frame}

\section{公式与环境}

\begin{frame}[t]{常见公式排版}
  典型的行间与对齐环境:
  \begin{equation}
    E=mc^2
  \end{equation}

  多行对齐(\texttt{aligned}):
  \begin{equation}
    \begin{aligned}
      a^2+b^2 &= c^2,\\
      \nabla\cdot \bm{E} &= \frac{\rho}{\varepsilon_0}.
    \end{aligned}
  \end{equation}
\end{frame}

% 交叉引用示例:图/表/公式
\begin{frame}[t]{交叉引用示例}
  本页演示三类引用:\figref{fig:single}、\tabref{tab:perf} 与 \eqrefc{eq:demo}。
  \vspace{0.5em}
  \begin{equation}
    a^2 + b^2 = c^2\,,\quad \label{eq:demo}
  \end{equation}
  其中 \eqrefc{eq:demo} 为示例公式标签。
\end{frame}

\section{使用教程}

\subsection{研究背景}
\subsubsection{三级标题}

\begin{frame}[c]{列表怎么用}
就是下面这么用,前面是个点点,原因是我在模板里自定义了,你可以修改,模板中搜索:列表前面的点点自定义
  \begin{itemize}
  \item 简捷易行
  \item 保留原文的格调
  \item[1] 你也可以用编号
  \item[2] 你也可以用编号
  \end{itemize}
还有这一种,也挺好看的:
  \begin{enumerate}
    \item 示例一
    \item 示例二
  \end{enumerate}
\end{frame}

% MATLAB 实时代码与输出环境示例
\begin{frame}[t,fragile]{MATLAB 代码与输出环境}
  lzu.sty 定义了 `matlabcode` 与 `matlaboutput` 环境,便于直接嵌入代码与命令行输出:
  \begin{matlabcode}
% 计算并绘图(示例)
x = 0:0.1:2*pi;
y = sin(x);
plot(x, y);
  \end{matlabcode}
  \begin{matlaboutput}
% 命令行输出示例
>> size(y)
ans =
     1   63
  \end{matlaboutput}
\end{frame}

% PDF 图片包含示例(若文件存在则显示,否则用占位框)
\begin{frame}[t,fragile]{PDF 图片包含示例}
  将 PDF 放到 `image/` 目录后,使用如下语句包含:
  \begin{block}{代码示例}
  \verb|\includegraphics[width=0.7\linewidth]{image/sample.pdf}|
  \end{block}
  若 PDF 含多页,支持选页:
  \begin{block}{选页示例}
  \verb|\includegraphics[page=2,width=0.7\linewidth]{image/sample.pdf}|
  \end{block}
\end{frame}

\begin{frame}
  \frametitle{每一页内容位置}
  内容老是居中,我想让它居上怎么办?看下一页。
\end{frame}

\begin{frame}[t]{每一页内容位置}
  这句话在上面了吧,原因是 frame 选项里加了 [t](顶部对齐),默认是 [c](居中)。
\end{frame}

\subsection{主要内容}

\begin{frame}
  \frametitle{图并排(单图)}
  \begin{figure}[H]
    \centering
    \includegraphics[width=0.30\textwidth]{source/lzu_logo.png}
    \caption{兰州大学(单图)\footnotesize 小字可以这样放在 caption 中}
    \label{fig_lzu_logo}
  \end{figure}
\end{frame}

\begin{frame}
  \frametitle{图并排(子图)}
  \begin{figure}[H]
    \centering
    \subfloat[图1]{\label{fig_lzus_0}\includegraphics[width=0.30\textwidth]{source/lzu_logo.png}}
    \hspace{1em}
    \subfloat[图2]{\label{fig_lzus_1}\includegraphics[width=0.30\textwidth]{source/lzu_logo.png}}\\
    \caption{兰州大学}
    \label{fig_lzus}
  \end{figure}
\end{frame}

\begin{frame}
  \frametitle{表格}
  \begin{table}[H]
    \centering
    \caption{纳米管参数(示例)}
    \begin{tabular}{cccccc}
        \toprule
        参数 & m  & n  & 原子数 & 内径 & 长度 \\
        \midrule
        二硫化钼纳米管 & 15 & 15 & 3420 & 2.3014nm & 11.85nm \\
        碳纳米管       & 16 & 6  & 1112 & 1.5424nm & 6.0nm   \\
        \bottomrule
    \end{tabular}
    \label{tbl_mcnt_nanotube_demo}
  \end{table}
\end{frame}

\begin{frame}
\frametitle{左右分栏}
去掉 \texttt{\textbackslash pause} 就不会分成两页了。
  \begin{columns}
  \column{0.6\textwidth}
    \begin{itemize}
    \item Ice Age
    \item The Hobbit
    \item The Great Gatsby
    \end{itemize}
  \pause
  \column{0.4\textwidth}
    \begin{itemize}
    \item 冰河世纪
    \item 霍比特人
    \item 了不起的盖茨比
    \end{itemize}
  \end{columns}
\end{frame}

\begin{frame}
  \frametitle{分步动画(逐步呈现)}
  \onslide<1>{
    第一步显示这一句话,第二步时消失。
  }
  \onslide<2-3>{
    \begin{block}{解析:}
      第二、三步显示这一句话。
    \end{block}
  }
  \onslide<3>{
    只有第三步显示这一句话。
  }
  更多 beamer 动画语法可参考 \parencite{partl2016}。
\end{frame}

\begin{frame}[t]{块环境(说明/结论)}
  \begin{block}{结论}
    使用 \textbf{块环境} 可以清晰强调结论或要点。
  \end{block}
  \begin{alertblock}{提示}
    在演示型文稿中,建议每页仅包含 1–2 个关键信息点。
  \end{alertblock}
\end{frame}

\section{参考文献}

\begin{frame}[t,allowframebreaks]{参考文献}
  \printbibliography
\end{frame}

\end{document}
