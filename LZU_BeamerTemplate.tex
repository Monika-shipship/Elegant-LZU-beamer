\documentclass[10pt,aspectratio=43]{beamer}

%%%%----- 宏包区 -----%%%%
\usepackage{lzu}
% 参考文献资源(由样式文件加载 biblatex)
\addbibresource{reference.bib}
\usepackage{xeCJK}
\usepackage{amsmath,amsfonts,amssymb,bm}
\usepackage{color}
\usepackage{graphicx,hyperref,url}
\usepackage{float}
\usepackage{subfig}
\usepackage{booktabs}

% 避免 hyperref 的 PDF 字符串告警(字号、粗体等)
\pdfstringdefDisableCommands{%%
  \def\fontsize#1#2{}%%
  \def\selectfont{}%%
  \def\textbf#1{#1}%%
  \def\parencite#1{}%%
}

% 降低非关键告警噪音(可按需调整)
\tracinglostchars=0
\hbadness=10000
\vbadness=10000
\hfuzz=50pt
\vfuzz=50pt

% 字体(保留原模板设置)
\setsansfont[Path=fonts/]{Helvetica}

% 进入每个 section 时自动插入“目录”页
\AtBeginSection[]
{
  \begin{frame}<beamer>
    \frametitle{\textbf{目录}}
    \tableofcontents[currentsection]
  \end{frame}
}

%%%%---- 基本信息 ----%%%%
\title[LZU Beamer 模板演示]{\fontsize{14pt}{18pt}\selectfont LZU Beamer 模板演示}
\subtitle{\fontsize{10pt}{14pt}\selectfont \textbf{排版示例:目录/图表/公式}}

\author[某某]{\textbf{某某} \\\medskip {\small 兰州大学}}
\institute[兰州大学]{兰州大学}
\date[\today]{\today}

\begin{document}

% 标题页
\begin{frame}
  \titlepage
\end{frame}

% 总目录(完整展示)
\begin{frame}
  \frametitle{\textbf{目录}}
  \tableofcontents
\end{frame}

\section{模板概览}

\begin{frame}[t]{为什么使用本模板}
  本模板在 SEU Beamer 基础上适配 LZU:
  \begin{itemize}
    \item 已集成主题配色、页眉/页脚、标题格式和背景图;
    \item 开箱即用,示例涵盖常见元素(图/表/公式/多栏排版);
    \item 使用 XeLaTeX 编译即可得到中文友好的排版效果;
    \item 保留 biblatex 支持,便于学术场景引用与生成参考文献。
  \end{itemize}
\end{frame}

\begin{frame}[t,fragile]{页面与结构建议}
  \begin{itemize}
    \item 使用 section/subsection 合理组织内容,目录页将自动更新;
    \item 每页聚焦 1–2 个关键信息点,避免过多文字;
    \item 图表配合简短说明,必要时给出来源和标签方便引用;
    \item 保持术语统一,跨页引用请使用 \verb|\\label|/\verb|\\ref|。
\end{itemize}
\end{frame}

\section{图表示例}

\begin{frame}[t]{单图示例}
  \begin{figure}[H]
    \centering
    \includegraphics[width=0.42\textwidth]{source/lzu_logo.png}
    \caption{单图示例(演示图片路径与缩放)}
    \label{fig:single}
  \end{figure}
\end{frame}

\begin{frame}[t]{并排子图}
  \begin{figure}[H]
    \centering
    \subfloat[子图A]{\includegraphics[width=0.38\textwidth]{source/lzu_logo.png}}\hspace{1em}
    \subfloat[子图B]{\includegraphics[width=0.38\textwidth]{source/lzu_title.png}}\\
    \caption{并排子图示例。参见 \figref{fig:single} 的单图示例。}
    \label{fig:subfig}
  \end{figure}
\end{frame}

\begin{frame}[t]{表格示例(booktabs)}
  \begin{table}[H]
    \centering
    \caption{性能对比示例表}
    \begin{tabular}{lccc}
      \toprule
      方法 & 指标A & 指标B & 指标C \\
      \midrule
      方法一 & 0.82 & 0.76 & 0.91 \\
      方法二 & 0.85 & 0.79 & 0.89 \\
      方法三 & 0.80 & 0.81 & 0.92 \\
      \bottomrule
    \end{tabular}
    \label{tab:perf}
  \end{table}
\end{frame}

\section{公式与环境}

\begin{frame}[t]{常见公式排版}
  典型的行间与对齐环境:
  \begin{equation*}
    E=mc^2
  \end{equation*}

  多行对齐(\texttt{aligned}):
  \begin{equation*}
    \begin{aligned}
      a^2+b^2 &= c^2,\\
      \nabla\cdot \bm{E} &= \frac{\rho}{\varepsilon_0}.
    \end{aligned}
  \end{equation*}
\end{frame}

\begin{frame}[t]{块环境(说明/结论)}
  \begin{block}{结论}
    使用 \textbf{块环境} 可以清晰强调结论或要点。
  \end{block}
  \begin{alertblock}{提示}
    在演示型文稿中,建议每页仅包含 1–2 个关键信息点。
  \end{alertblock}
\end{frame}

\section{参考文献}

\begin{frame}[t,allowframebreaks]{参考文献}
  \printbibliography
\end{frame}

\end{document}
